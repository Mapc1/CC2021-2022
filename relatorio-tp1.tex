\documentclass{article}

\usepackage{authblk}
\usepackage{graphicx}
\usepackage{multicol}

\title{
    Comunicação por Computador \\
    \large{Trabalho Prático 1}
}
\author{
    Marco Costa A93283
}
\date{27 de outubro de 2021}
\affil{
    Departamento de Informática \\
    Universidade do Minho
}

\begin{document}
    \begin{titlepage}
        \maketitle
    \end{titlepage}
    \section*{Respostas}
        \subsection*{1}
            {
                \centering
                \includegraphics[width=12cm]{images/lol.png}
                \includegraphics[width=12cm]{images/lol-grilo.png}
                \par
            }
                Como podemos ver nas imagens o nó \textit{Grilo}, que no \textit{CORE} lhe foi dada
            uma conexão com 10\% de duplicação e 5\% de packet loss, de 100 pacotes enviados apenas recebeu
            94 mais 11 pacotes duplicados. Isto contrasta com o \textit{Portatil1} que com uma conexão boa conseguiu
            receber os 100 pacotes de resposta sem qualquer duplicação ou perda destes.\par

                Mais interessante é o resultado destes problemas de conexão no \textit{rtt(round-trip time)} em que se pode verificar
            que o \textit{Grilo} teve valores significativamente mais elevados que a sua contraparte. Apesar disto o \textit{Grilo} demorou
            menos 2101ms a executar.
        \subsection*{2}
            {
                \centering
                \includegraphics[width=12cm]{images/ftp-wireshark.png}
                \par
            }
        \subsection*{3}
            {
                \centering
                \includegraphics[width=12cm]{images/tftp-wireshark.png}
                \par
            }
                O protocolo tftp mostra-se bem mais simples que o ftp tendo sido apenas trocados 3 pacotes.
            A conexão é iniciada com um \textit{Read Request}, o servidor envia o ficheiro e, por fim, o portátil responde com um \textit{Acknoulegment}.
        \subsection*{4}
                Após o uso dos quatro programas podemos concluir que o mais seguro é o sftp pelo seu uso do protocolo ssh que encripta toda a comunicação entre despositivos.
            Em termos de simplicidade http é o melhor devido ao facto de que só é necessário enviar um pacote \textit{GET} e o servidor retorna o ficheiro.
\end{document}
